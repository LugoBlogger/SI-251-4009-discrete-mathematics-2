\documentclass[a4paper,12pt]{article}
\usepackage{fancyhdr}
\usepackage{graphicx}
\usepackage{amsmath}
\usepackage[nomarginpar, top=1cm, left=1cm, right=1.2cm, bottom=1cm]{geometry}

\usepackage[scaled]{helvet}
\usepackage[T1]{fontenc}
\renewcommand{\familydefault}{\sfdefault}
\usepackage{mathpazo}
\usepackage[bb=pazo]{mathalpha}
\usepackage{tabularx}
\usepackage{multicol}

\pagenumbering{gobble}

\newcommand{\ITKheader}[3]{%
  \begin{center}
    \begin{tabular}{m{0.25\textwidth}>{\centering\arraybackslash}m{0.695\linewidth} }
      \hline
      \includegraphics[width=0.25\textwidth]{./Logo_ITK.png}
      & \large\textbf{MID-TERM EXAM} \par
        ACADEMIC YEAR #1 - #2 SEMESTER \par
        #3 STUDY PROGRAMME \par 
        INSTITUT TEKNOLOGI KALIMANTAN \par 
        \\ \hline
    \end{tabular}
  \end{center}
}


\newcommand{\courseDetail}[6]{%
\begin{center}
  \bgroup
  \def\arraystretch{1.5}
  \begin{tabular}{ll|ll}
    Course Name       &: #1
      & Duration  &: #4 \\ 
    Number of Credits &: #2             
      & Date      &: #5 \\ 
    Lecturer          &: #3
      & Exam type &: #6 \\ 
  \end{tabular}
  \egroup
\end{center}
}


\begin{document}

\ITKheader{2025/2026}{ODD}{INFORMATION SYSTEMS}

\courseDetail{Discrete Mathematics 2}%
  {3 credits}%
  {Henokh Lugo Hariyanto, M.Sc.}%
  {120 minutes}%
  {Friday / October 17th, 2025}%
  {Open an A4 cheat sheet}

\bgroup
\noindent
\setlength{\fboxsep}{8pt}  
\boxed{
  \textbf{SOAL PAKET C}%
}
\egroup

\vspace{12pt}

\textbf{Kerjakan soal-soal berikut dari yang paling bisa dikerjakan terlebih dahulu.}

\begin{enumerate}
  \item Diberikan ilustrasi pergerakan di ruang tiga dimensi dengan 
  sumbu-$x$, $y$, dan $z$.
  
  \begin{center}
  \includegraphics[width=0.5\linewidth]{./c-type-prob-1.png} 
  \end{center}
  Berapa banyak cara untuk berjalan di ruang tiga dimensi tersebut dari titik 
  $(0, 0, 0)$, ke titik $(4, 3, 5)$ dengan cara mengambil satu unit langkah
  bilangan bulat positif searah sumbu-$x$, satu unit langkah 
  bilangan bulat positif searah sumbu-$y$, atau satu unit langkah bilangan 
  bilangan bulat positif searah sumbu-$z$? 
  [\textit{Catatan}: Pergerakan searah sumbu-$x$, $y$, atau $z$ negatif tidak
  diperbolehkan].

  \item Gunakan teorema binomial untuk mencari koefisien dari $x^a y^b$
    dari ekspansi $(5x^2 + 2y^3)^6$, dengan 
    \begin{enumerate}
      \item $a = 6$, $b = 9$. 
      \item $a = 2$, $b = 15$.
      \item $a = 3$, $b = 12$. 
      \item $a = 12$, $b = 0$.
      \item $a = 8$, $b = 9$.
    \end{enumerate}

  \item Suatu relasi $R$ pada himpunan $A$ dikatakan anti-simetrik jika 
    diberikan $(a, b) \in R$ dan $(b, a) \in R$, kita memiliki $a = b$, 
    untuk setiap $a \in A$ dan $b \in A$. Tentukan apakah relasi $R$
    berikut pada himpunan seluruh bilangan riil bersifat anti-simetrik 
    dengan $(x, y) \in R$ diberikan sebagai
    \begin{multicols}{2}
    \begin{enumerate}
      \item $x + y = 0$.
      \item $x = \pm y$.
      \item $x - y$ adalah bilangan rasional.
      \item $x = 2y$.
      \item $xy \geq 0$.
      \item $xy = 0$.
      \item $x = 1$.
      \item $x = 1$ atau $y = 1$
    \end{enumerate}
    \end{multicols}

  \vfill
  \newpage
  \item Suatu relasi $R$ dapat diwakilkan dengan matriks $\mathbf{M}_R = [m_{ij}]$
    yang mana elemen matriks $(i, j)$ diberi nilai 1 jika terdapat 
    anggota $(a_i, b_j)$ di $R$, sedangkan diberi nilai 0 jika tidak terdapat
    anggota $(a_i, b_j)$ di $R$.   

    Carilah daftar anggota pasangan $(a, b)$ yang dapat diperoleh dari matriks
    berikut untuk relasi $R$ pada himpunan $\{1, 2, 3, 4\}$
    \begin{multicols}{3}
    \begin{enumerate}
      \item $\begin{pmatrix} 
          1 & 1 & 0 & 1 \\ 1 & 1 & 1 & 0 \\ 0 & 1 & 0 & 1 \\ 1 & 0 & 1 & 1
        \end{pmatrix}$
      \item $\begin{pmatrix}
          1 & 0 & 0 & 0 \\ 1 & 1 & 0 & 0 \\ 1 & 0 & 0 & 1 \\ 1 & 0 & 1 & 1
      \ \end{pmatrix}$
      \item $\begin{pmatrix}
          1 & 0 & 1 & 0 \\ 0 & 1 & 0 & 1 \\ 1 & 0 & 1 & 0 \\ 0 & 1 & 0 & 1
        \end{pmatrix}$
    \end{enumerate}
    \end{multicols}

\end{enumerate}

\vfill

\begin{center}
  \textbf{Selamat mengerjakan}

  \vspace{12pt}
  
  \textit{"Future success isn't solely determined by academic grades but 
    by the lasting knowledge and understanding one retains beyond 
    formal education"}
\end{center}

\newpage

\ITKheader{2025/2026}{ODD}{INFORMATION SYSTEMS}

\courseDetail{Discrete Mathematics 2}%
  {3 credits}%
  {Henokh Lugo Hariyanto, M.Sc.}%
  {120 minutes}%
  {Friday / October 17th, 2025}%
  {Open an A4 cheat sheet}

\bgroup
\noindent
\setlength{\fboxsep}{8pt}  
\boxed{
  \textbf{SOAL PAKET D}%
}
\egroup

\vspace{12pt}

\textbf{Kerjakan soal-soal berikut dari yang paling bisa dikerjakan terlebih dahulu.}

\begin{enumerate}
  \item Seorang \textit{youtuber} menyimpan koleksi 40 video yang 
    dia buat ke dalam ke dalam 4 \textit{hard disk} eksternal 
    dengan kapasitas 10 video untuk tiap \textit{hard disk} eksternal. 
    Berapa banyak cara yang dapat \textit{youtuber} lakukan untuk 
    mendistribusikan video tersebut jika 
    \begin{enumerate}
      \item setiap \textit{hard disk} eksternal diberi label angka berurut, 
        sehingga setiap \textit{hard disk} eksternal dapat dibedakan?
      \item semua \textit{hard disk} eksternal tidak diberi label, sehingga
        setiap \textit{hard disk} eksternal tidak dapat dibedakan?
    \end{enumerate}

  \item Diberikan definisi \textbf{permutasi}-$r$ \textbf{melingkar}
    dari $n$ orang adalah banyaknya cara mendudukan $r$ orang 
    (dengan $ r \leq n$) di suatu meja bundar. Setiap cara mendudukan 
    $r$ orang tersebut dianggap serupa (terhitung lebih dari satu kali) jika 
    kita bisa memperoleh cara tersebut dengan merotasikan meja bundar tersebut.
    Berikut contoh cara mendudukan 3 orang namun dihitung satu cara

    \begin{center}
      \includegraphics[width=0.7\linewidth]{./d-type-prob-2.png}
    \end{center}

    Carilah nilai permutasi-3 melingkar untuk 5 orang.

  \item Suatu relasi $R$ pada himpunan $A$ dikatakan anti-simetrik jika 
    diberikan $(a, b) \in R$ dan $(b, a) \in R$, kita memiliki $a = b$, 
    untuk setiap $a \in A$ dan $b \in A$. Tentukan apakah relasi $R$
    berikut pada himpunan $\{1, 2, 3, 4\}$ bersifat anti-simetrik 
    dengan $(x, y) \in R$ diberikan sebagai
    \begin{multicols}{2}
    \begin{enumerate}
      \item $\{(2, 2), (2, 3), (2, 4), (3, 2), (3, 3), (3, 4)\}$
      \item $\{(1, 1), (1, 2), (2, 1), (2, 2), (3, 3), (4, 4)\}$
      \item $\{(2, 4), (4, 2)\}$
      \item $\{(1, 2), (2, 3), (3, 4)\}$
      \item $\{(1, 1), (2, 2), (3, 3), (4, 4)\}$
      \item $\{(1, 3), (1, 4), (2, 3), (2, 4), (3, 1), (3, 4)\}$
    \end{enumerate}
    \end{multicols}

  \vfill \newpage

  \item Suatu relasi $R$ dapat diwakilkan dengan matriks $\mathbf{M}_R = [m_{ij}]$
    yang mana elemen matriks $(i, j)$ diberi nilai 1 jika terdapat 
    anggota $(a_i, b_j)$ di $R$, sedangkan diberi nilai 0 jika tidak terdapat
    anggota $(a_i, b_j)$ di $R$.   

    Carilah daftar anggota pasangan $(a, b)$ yang dapat diperoleh dari matriks
    berikut untuk relasi $R$ pada himpunan $\{1, 2, 3, 4\}$
    \begin{multicols}{3}
    \begin{enumerate}
      \item $\begin{pmatrix} 
          1 & 1 & 0 & 1 \\ 1 & 0 & 1 & 0 \\ 0 & 1 & 1 & 1 \\ 1 & 0 & 1 & 1
        \end{pmatrix}$
      \item $\begin{pmatrix}
          1 & 1 & 1 & 0 \\ 0 & 1 & 0 & 0 \\ 0 & 0 & 1 & 1 \\ 1 & 0 & 0 & 1
      \ \end{pmatrix}$
      \item $\begin{pmatrix}
          0 & 1 & 0 & 1 \\ 1 & 0 & 1 & 0 \\ 0 & 1 & 0 & 1 \\ 1 & 0 & 1 & 0
        \end{pmatrix}$
    \end{enumerate}
    \end{multicols}

\end{enumerate}

\vfill

\begin{center}
  \textbf{Selamat mengerjakan}

  \vspace{12pt}
  
  \textit{"Future success isn't solely determined by academic grades but 
    by the lasting knowledge and understanding one retains beyond 
    formal education"}
\end{center}

\end{document}