\documentclass[a4paper,12pt]{article}
\usepackage{fancyhdr}
\usepackage{graphicx}
\usepackage{amsmath}
\usepackage[nomarginpar, top=1cm, left=1cm, right=1.2cm, bottom=1cm]{geometry}

\usepackage[scaled]{helvet}
\usepackage[T1]{fontenc}
\renewcommand{\familydefault}{\sfdefault}
\usepackage{mathpazo}
\usepackage[bb=pazo]{mathalpha}
\usepackage{tabularx}

\pagenumbering{gobble}

\newcommand{\ITKheader}[3]{%
  \begin{center}
    \begin{tabular}{m{0.25\textwidth}>{\centering\arraybackslash}m{0.695\linewidth}}
      \hline
      \includegraphics[width=0.25\textwidth]{./Logo_ITK.png}
      & \large\textbf{MID-TERM EXAM} \par
        ACADEMIC YEAR #1 - #2 SEMESTER \par
        #3 STUDY PROGRAMME \par 
        INSTITUT TEKNOLOGI KALIMANTAN \par 
        \\ \hline
    \end{tabular}
  \end{center}
}


\newcommand{\courseDetail}[6]{%
\begin{center}
  \bgroup
  \def\arraystretch{1.5}
  \begin{tabular}{ll|ll}
    Course Name       &: #1
      & Duration  &: #4 \\ 
    Number of Credits &: #2             
      & Date      &: #5 \\
    Lecturer          &: #3
      & Exam type &: #6 \\
  \end{tabular}
  \egroup
\end{center}
}


\begin{document}

\ITKheader{2025/2026}{ODD}{INFORMATION SYSTEMS}

\courseDetail{Discrete Mathematics 2}%
  {3 credits}%
  {Henokh Lugo Hariyanto, M.Sc.}%
  {120 minutes}%
  {Wednesday / October 15th, 2025}%
  {Open an A4 cheat sheet}

\bgroup
\noindent
\setlength{\fboxsep}{8pt}  
\boxed{
  \textbf{SOAL PAKET A}%
}
\egroup

\vspace{12pt}

\textbf{Kerjakan soal-soal berikut dari yang paling bisa dikerjakan terlebih dahulu.}

\begin{enumerate}
  \item Ada berapa banyak cara untuk mendistribusikan lima bola ke tujuh 
    kotak jika setiap kotak harus memiliki paling banyak satu bola dan jika 
    \begin{enumerate}
      \item bola dan kotak memiliki label yang berbeda-beda?
      \item bola memiliki label yang berbeda-beda, namun kotak tidak memiliki label?
      \item bola tidak memiliki label, namun kotak memiliki label yang berbeda-beda?
      \item bola dan kotak tidak memiliki label?
    \end{enumerate}
  
  \item Seluruh permutasi dari $\{1, 2, 3, \ldots, n\}$ dapat disusun berdasarkan
    \textit{lexicographic ordering}. Susunan ini mengisyaratkan bahwa permutasi 
    $a_1 a_2 \cdots a_n$ memiliki urutan lebih awal dari permutasi 
    $b_1 b_2 \cdots b_n$, jika untuk $k$ tertentu, dengan $1 \leq k \leq n$, 
    kita memiliki 
    \begin{align*}
      \left\{
      \begin{aligned}
        &a_1 = b_1, \\
        &a_2 = b_2, \\
        &\quad \vdots \\
        &a_{k-1} = b_{k-1}, \\ 
        &a_{k} < b_k
      \end{aligned}
      \right.
    \end{align*}
  
    Misalkan nama suatu berkas di suatu komputer tersusun atas 
    tiga huruf besar dan diikuti satu digit. Huruf besar tersebut dapat 
    berupa A, B, atau C, dan digit yang dimaksud adalah 1 atau 2. 
    Cari\-lah seluruh nama berkas tersebut dan urutkan berdasarkan 
    \textit{lexicographic order} yang mana kita menyusun urutan huruf 
    sesuai urutan abjad.

  \item Pangkat $n$ dari suatu relasi $R$ pada himpunan $A$ didefinisikan 
    secara rekursif sebagai berikut
    \begin{align*}
      R^1 = R \quad \text{ dan }\quad R^{n+1} = R^n \circ R
    \end{align*}
    untuk $n = 1, 2, 3, \ldots$.
  
    Misalkan $R$ adalah relasi pada himpunan $\{1, 2, 3, 4, 5\}$
    yang tersusun atas pasangan terurut $(1, 1)$, $(1, 2)$, $(1, 3)$, 
    $(2, 3)$, $(2, 4)$, $(3, 1)$, $(3, 4)$, $(3, 5)$, 
    $(4, 2)$, $(4, 5)$, $(5, 1)$, $(5, 2)$, dan $(5, 4)$. Carilah
    \begin{enumerate}
      \item $R^2$
      \item $R^4$
    \end{enumerate}

  \vfill
  \newpage
  \item Suatu relasi $R$ dapat diwakilkan dengan matriks $\mathbf{M}_R = [m_{ij}]$
    yang mana elemen matriks $(i, j)$ diberi nilai 1 jika terdapat 
    anggota $(a_i, b_j)$ di $R$, sedangkan diberi nilai 0 jika tidak terdapat
    anggota $(a_i, b_j)$ di $R$.   

    Nyatakan relasi $R$ pada himpunan $A = \{1, 2, 3, 4\}$ berikut 
    \begin{enumerate}
      \item $\{(1, 2), (1, 3), (1, 4), (2, 3), (2, 4), (3, 4)\}$
      \item $\{(1, 1), (1, 4), (2, 2), (3, 3), (4, 1)\}$ 
    \end{enumerate}
    ke dalam bentuk matriks.

\end{enumerate}

\vfill

\begin{center}
  \textbf{Selamat mengerjakan}

  \vspace{12pt}
  
  \textit{"Future success isn't solely determined by academic grades but 
    by the lasting knowledge and understanding one retains beyond 
    formal education"}
\end{center}

\newpage

\ITKheader{2025/2026}{ODD}{INFORMATION SYSTEMS}

\courseDetail{Discrete Mathematics 2}%
  {3 credits}%
  {Henokh Lugo Hariyanto, M.Sc.}%
  {120 minutes}%
  {Wednesday / October 15th, 2025}%
  {Open an A4 cheat sheet}

\bgroup
\noindent
\setlength{\fboxsep}{8pt}  
\boxed{
  \textbf{SOAL PAKET B}%
}
\egroup

\vspace{12pt}

\textbf{Kerjakan soal-soal berikut dari yang paling bisa dikerjakan terlebih dahulu.}

\begin{enumerate}
  \item Ada berapa banyak cara untuk mendistribusikan lima bola ke tiga 
    kotak jika setiap kotak harus memiliki paling sedikit satu bola dan jika 
    \begin{enumerate}
      \item bola dan kotak memiliki label yang berbeda-beda?
      \item bola memiliki label yang berbeda-beda, namun kotak tidak memiliki label?
      \item bola tidak memiliki label, namun kotak memiliki label yang berbeda-beda?
      \item bola dan kotak tidak memiliki label?
    \end{enumerate}

  \item Seluruh permutasi dari $\{1, 2, 3, \ldots, n\}$ dapat disusun berdasarkan
    \textit{lexicographic ordering}. Susunan ini mengisyaratkan bahwa permutasi 
    $a_1 a_2 \cdots a_n$ memiliki urutan lebih awal dari permutasi 
    $b_1 b_2 \cdots b_n$, jika untuk $k$ tertentu, dengan $1 \leq k \leq n$, 
    kita memiliki 
    \begin{align*}
      \left\{
      \begin{aligned}
        &a_1 = b_1, \\
        &a_2 = b_2, \\
        &\quad \vdots \\
        &a_{k-1} = b_{k-1}, \\ 
        &a_{k} < b_k
      \end{aligned}
      \right.
    \end{align*}
  
    Misalkan nama suatu direktori di suatu komputer tersusun atas 
    tiga digit dan diikuti dua huruf kecil. Tiga digit tersebut adalah 
    0, 1, atau 2, dan huruf kecil tersebut adalah a atau b.
    Cari\-lah seluruh nama berkas tersebut dan urutkan berdasarkan 
    \textit{lexicographic order} yang mana kita menyusun urutan huruf 
    sesuai urutan abjad.

  \item Pangkat $n$ dari suatu relasi $R$ pada himpunan $A$ didefinisikan 
    secara rekursif sebagai berikut
    \begin{align*}
      R^1 = R \quad \text{ dan }\quad R^{n+1} = R^n \circ R
    \end{align*}
    untuk $n = 1, 2, 3, \ldots$.
  
    Misalkan $R$ adalah relasi pada himpunan $\{1, 2, 3, 4, 5\}$
    yang tersusun atas pasangan terurut    
    
    $\{(1, 1), (1, 3), (1, 5), (2, 1), (2, 4), (2, 5), (3, 1), (3, 2), (4, 2), (4, 3), (4, 5), (5, 3), (5, 4)\}$. Carilah
    \begin{enumerate}
      \item $R^3$
      \item $R^5$
    \end{enumerate}

  \vfill
  \newpage

  \item Suatu relasi $R$ dapat diwakilkan dengan matriks $\mathbf{M}_R = [m_{ij}]$
    yang mana elemen matriks $(i, j)$ diberi nilai 1 jika terdapat 
    anggota $(a_i, b_j)$ di $R$, sedangkan diberi nilai 0 jika tidak terdapat
    anggota $(a_i, b_j)$ di $R$.   

    Nyatakan relasi $R$ pada himpunan $A = \{1, 2, 3, 4\}$ berikut 
    \begin{enumerate}
      \item $\{(1, 2), (1, 3), (1, 4), (2, 1), (2, 3), (2, 4), (3, 1),
        (3, 2), (3, 4), (4, 1), (4, 2), (4, 3)\}$
      \item $\{(2, 4), (3, 1), (3, 2), (3, 4)\}$ 
    \end{enumerate}
    ke dalam bentuk matriks.

\end{enumerate}

\vfill

\begin{center}
  \textbf{Selamat mengerjakan}

  \vspace{12pt}
  
  \textit{"Future success isn't solely determined by academic grades but 
    by the lasting knowledge and understanding one retains beyond 
    formal education"}
\end{center}

\end{document}